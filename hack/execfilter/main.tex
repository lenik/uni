\documentclass[hyperref, bookmark]{z-article}
\usepackage{z-filetree}
\usepackage{verbatim}
\usepackage{environ}
\usepackage{fancyvrb}
\usepackage{mdframed}

\newcommand\code[1]{\texttt{#1}}

\DefineVerbatimEnvironment{console}{Verbatim}{
  frame=lines,
  framerule=0.5mm,
  rulecolor=\color{red},
  label=CONSOLE
}

\DefineVerbatimEnvironment{codeblk}{Verbatim}{
  frame=single,
  framerule=0.5mm,
  rulecolor=\color{green!20!black},
  numbers=left
}

\newcommand\syntax[1]{
  \begin{mdframed}[
      linecolor=brown,
      linewidth=2pt,
    ]
    #1
  \end{mdframed}
}

\newcommand\xfilt{\textbf{execfilter}}
\newcommand\configdir{\texttt{/etc/execfilter.d/}}

\setlength{\parskip}{\baselineskip}

\title {execfilter \\ Execution Filter For Linux }
\author{Lenik \\ \boDz TopCroak Software}

\begin{document}
\maketitle
\tableofcontents

\clearpage

\section {Quick Start}

There are many applications under Linux which may spawn vulnerable processes to
be hacked in.  The trivial Unix execution bits on the file won't let you to
configure them specifically according to the needs of different applications.


\xfilt is just born for this task. It's a shared library dynamically inserted
before libc, where the most process creation calls are made.  To use this
utility, you need to write some configurations under \configdir, one may look
like this:

\begin{codeblk}
  # file: /etc/execfilter.d/warcraft
  # Prevent the game warcraft to launch the shell.
  for /usr/bin/warcraft
      deny /bin/sh
\end{codeblk}

That's it! \xfilt will read all files (but without any sub-directory) in
\configdir, parse and analyze them, if the target executable the current process
wanted to spawn is listed in any of the denied statement, the execution will be
immediately failed.

\section {Install}

To install \xfilt, you need to include the \boDz sources list in your apt
config:

\begin{console}
  ## You need to be root to run this command:

  # echo "deb http://deb.lenik.net/ unstable/" >> /etc/apt/sources.list
  # apt-get update
  # apt-get install execfilter
\end{console}

Or, you may download and compile yourself:

\begin{console}
  $ cd /home/my/build/dirs
  $ wet http://....../execfilter.tar.bz2
  $ tar xJf execfilter-1.0.5.tar.bz2
  $ cd execfilter-1.0.5
  $ ./configure
  $  (...)
  $ make
  $ make install
\end{console}

This will build the source code and install the binaries to \code{/usr/local}
by default.

The installed files are structured like:

\def\filetreescale{0.7}
\begin{tikzpicture}[
    scale=\filetreescale,
    every node/.style={transform shape},
    grow via three points={one child at (0.5,-0.7) and
      two children at (0.5,-0.7) and (0.5,-1.4)},
    edge from parent path={(\tikzparentnode.south) |- (\tikzchildnode.west)}]
    \path node[dir] {\emph{«setupdir»}}
    child{
        node[dir] {usr\fileflag{/}}
        child{
            node[dir] {share\fileflag{/}}
            child{
                node[dir] {setup\fileflag{/}}
                child{
                    node[dir] {jsrcutils\fileflag{/}}
                    child{ node[executable] {postinst\fileflag{*}} }
                    child{ node[executable] {prerm\fileflag{*}} }
                }
            }
        }
    };
\end{tikzpicture}


\section {Configuration}

\xfilt use configuration files to describe rules applied to each
application. These files are placed in \configdir directory. The loading order
is arbitrary, as there is no dependency requirement in the rule definitions.

Empty lines and comment lines which starts with ``\#'' are ignored.  Each statement
is terminated by a new-line character.

A single config file may contains several sections, each section describes
executing rules applied to the caller process.  If a target program is denied by
a parent process, all child processes are automatically inherited the denied
rule.

Each statement can be on of the following:

\subsection {for statement}

\syntax{
  ``for'' \textit{<APP-PATH>}
}

The \code{for} statement specifies which application the following statements
will be applied to. The \textit{APP-PATH} should be the absolute path without
any quote char (' or ").  The trailing space is ignored, so you can't define any
rule for filenames with extra space in the end.

\begin{codeblk}
  for APP1
  ...
  ... (statements applied to APP1)
  ...
  for APP2
  ...
  ... (statements applied to APP2)
  ...
  for APP1
  ...
\end{codeblk}

The same application may be occurred in multiple sections in a single config
file.  They will be merged by the parser, thus allows you to group rules in your
own manner.

During \xfilt matching the caller process to the application path specified in
the \code{for} clause, the pathnames are canonicalized at the first.  So you can
write symbolic links and relative paths here. But for security consideration,
you should always use absolute paths here.


\subsection {default statement}

The \code{default} statement defines the default behavior if no rule matches the
execution target.

\syntax {
  ``default'' \{ ``allow'' | ``deny'' \}
}

There are only two forms: \code{default allow} and \code{default deny}.

When you omitted the \code{default} statement, the default behavior is deny
default.

The default behavior is not inherited. That is, when there are multiple rules
matching the ancestors of the caller process, each matching is evaluated first,
if any one is denied, the result is denied.

\subsection {allow/deny statement}

\syntax {
  \{ ``allow''| ``deny'' \} \textit{<TARGET-SPEC>}
}

\code{allow} and \code{deny} statements define how \xfilt handles the execution target.

\begin{codeblk}
  for /usr/bin/warcraft
    default deny
    allow /sbin/ifconfig

  for /usr/bin/btdownloader
    deny /sbin/pm-suspend
    deny /sbin/shutdown
\end{codeblk}

This example shows, for application \code{/usr/bin/warcraft}, the execution of
\code{/sbin/ifconfig} is allowed, because it may need \code{ifconfig} to start
network playing mode, and everything else is denied.  However,
\code{/usr/bin/btdownloader} uses an oppsite default mode, everything except
\code{/sbin/pm-suspend} and \code{/sbin/shutdown} are allowed, this is intended
to prevent from suspend or shutdown the machine after the downloader completed
its tasks.

The \textit{TARGET-SPEC} specify the execution target, like the
\code{for}-clause, it can be symbolic link name and relative path names, they
are later be canonicalized by \xfilt. The \textit{TARGET-SPEC} names are also be
finding against the PATH environ, so you can omit the ``/bin'', ``/usr/bin''
parts if they are in the system PATH.  But for security considerations, you
should use absolute pathnames whenever possible.

\section {Extension}

\xfilt supports two types of extension: configuration decoder and the system
gate. A decoder is a library function to transform data bytes. \xfilt only
supports fixed-size decoder, i.e., the size of the input bytes must match the
size of the output bytes. Configurations in different extensions can be encoded
in different way, to setup a specific format, you need to add the following line
to \code{\configdir/filter.ini}:

\syntax{
  ``format''
  \textit{<extension-name>}
  \textit{<library-name>}
  \textit{<function-name>}
}

The \textit{extension-name} is the last word of the filename without the dot
(`.'), it's case-sensitive on a case-sensitive file system. The
\textit{library-name} is the basename of the library, which will be searched
automatically against library-path. If the library isn't installed in
\code{/usr/lib} or somewhere known by the system, you should use the absolute
pathname here.

The \textit{function-name} is the function name defined in the library. It must
be defined as \code{fn(char *buf, size\_t size)}, where \code{buf} is used for
both input and output, and the \code{size} specified the size of the \code{buf}.

The system gate extension is used to dynamically determine whether \xfilt is
activated.

\syntax {
  ``gate''
  \textit{<library-name>}
  \textit{<function-name>}
}

The meaning of \textit{library-name} and \textit{function-name} are the same as
above. If the return value of the system gate function is zero, \xfilt will be
deactivated, and all executions are just pass thru to the original system calls.

\section {Diagnostics}

In case of problems, check the following guide:

\xfilt will update the \code{/etc/ld.so.preload} file which contains the
preloaded library list, to include \xfilt itself. Occasionally, the file might
be chmod to be unreadble from everyone.  This will cause \xfilt not work at
all. To fix this problem, just run \code{sudo chmod +r /etc/ld.so.preload}.

There maybe undiscovered bugs in the \xfilt implementation, which may cause the
system unstable.  To determine whether it's a bug of \xfilt or something else,
the \xfilt package contains a helper stub module.  Try to replace the
\code{libexecfilter.so} with \code{libefstub.so} in \code{/etc/ld.so.preload}
and check if the problem still exists.  If the problem is gone, it could be
something wrong within \xfilt.  Please tell me (me (at) lenik.net) when you are
in such cases.

\subsection {Set-UID applications}

Because \xfilt is implemented as a preloaded shared object, it could be ignored
in set-uid applications if the set-uid bit isn't set on the \code{.so} file.

By default, \xfilt installer will set the suid bit on the
\code{libexecfilter.so} file, however, some programs may eventually reset the
suid bit. You may notice the following warning when it happens:

\begin{console}
  ERROR: ld.so: object 'libexecfilter.so' from /etc/ld.so.preload
  cannot be preloaded: ignored.
\end{console}

Set the suid bit again to remove the warning:

\begin{console}
  # chmod +xs /usr/lib/libexecfilter.so
\end{console}

\appendix
%\section {索引}
%    \listoftables

\end{document}
