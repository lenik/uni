\newcommand\zclass{z-CLASS*}

\section{使用}

    z-latex 为常用的几个文档类添加了封装,如 book 对应为 z-book (以下简称在 \zclass)。 声明文档为 \zclass 的变体:

    \begin{itemize}
        \item \texttt { \textbackslash documentclass\{z-article\} }
        \item \texttt { \textbackslash documentclass\{z-book\} }
        \item \texttt { \textbackslash documentclass\{z-beamer\} }
    \end{itemize}

\section{中文支持}

    \zclass 的文档需要用 xelatex 编译。
    在 \LaTeX 中使用中文有很多麻烦的地方,比如源文件编码,字体设置,中文
    排版等等。\zclass 把中文支持作为一个基本选项,通过传递参数:

    \texttt { \textbackslash documentclass[CJK]\{\zclass\} }

    将导入和中文相关的 \LaTeX 包,并设置如字体、间距等参数的默认值。

\section{兼容性}

    \zclass 在扩展基本文档类的同时还引入了一些必要的包,
    大部分文档类选项都原分不动的传递给基本文档类了,因此通常不会有兼容性问题。

    但由于有些包的顺序有严格要求(常见的比较容易出错的包如
        \texttt{colorx},
        \texttt{hyperref},
        \texttt{fontspec}
    等),因此可能会导致无法加载一些包。

    当发生包冲突时,唯一的方法可能还是把 \zclass 模板的源文件复制粘帖到
    你的文档中,然后手工调整包的顺序。
